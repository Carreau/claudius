% arara: indent: {overwrite: yes, modifylinebreaks: yes}
% arara: pdflatex: {draft: yes, shell : yes}
% arara: bibtex
% arara: pdflatex: {draft: yes, shell : yes}
% arara: pdflatex: {shell : yes}
% arara: clean: { extensions: [ log ], files: [ indent ] }
% arara: clean: { extensions: [ aux, bak, log, toc, out ]}

\documentclass[12pt,a4paper]{article}


% Usepackages =============================================
% Standard et geometrie -----------------------------------
\usepackage[utf8]{inputenc}
\usepackage[T1]{fontenc}
\usepackage[english]{babel}
\usepackage[margin=2cm]{geometry}

% Maths ---------------------------------------------------
\usepackage{amsmath,amssymb,amsfonts,amsthm}
\usepackage{mathtools,mathrsfs}

%Liens et dessins -----------------------------------------
\usepackage[colorlinks]{hyperref}
\usepackage{tikz}


% Couleurs ================================================
% definecolor ---------------------------------------------
\definecolor{coGB}{HTML}{1F77B4}
\definecolor{coOr}{HTML}{FF7F0E}
\definecolor{coWG}{HTML}{2CA02C}
\definecolor{coAR}{HTML}{D62728}
\definecolor{cMBP}{HTML}{9467BD}
\definecolor{coSM}{HTML}{8C564B}
\definecolor{coPP}{HTML}{E377C2}
\definecolor{coTG}{HTML}{7F7F7F}
\definecolor{coAG}{HTML}{BCBD22}
\definecolor{coBB}{HTML}{17BECF}
% hypersetup ----------------------------------------------
\hypersetup{
    colorlinks = True,
    linkcolor  = coAR,
    citecolor  = coWG,
    urlcolor   = coGB,
}


% Maths ===================================================
% mathbb --------------------------------------------------
\newcommand{\bbA}{\mathbb{A}}
\newcommand{\bbC}{\mathbb{C}}
\newcommand{\bbD}{\mathbb{D}}
\newcommand{\bbN}{\mathbb{N}}
\newcommand{\bbR}{\mathbb{R}}
\newcommand{\bbS}{\mathbb{S}}
\newcommand{\bbT}{\mathbb{T}}
\newcommand{\bbZ}{\mathbb{Z}}

% mathcal --------------------------------------------------
\newcommand{\calD}{\mathcal{D}}
\newcommand{\calV}{\mathcal{V}}

% mathscr --------------------------------------------------
\newcommand{\scrC}{\mathscr{C}}

% mathrm --------------------------------------------------
\newcommand{\rmL}{\mathrm{L}}
\newcommand{\rmH}{\mathrm{H}}

% Constants -----------------------------------------------
\newcommand{\ex}{\mathsf{e}}
\newcommand{\im}{\mathsf{i}}

% Fontions ------------------------------------------------
\newcommand{\bJ}{\mathop{}\!\mathsf{J}}
\newcommand{\bY}{\mathop{}\!\mathsf{Y}}
\newcommand{\Hu}{\mathop{}\!\mathsf{H}^{(1)}}
\newcommand{\Hd}{\mathop{}\!\mathsf{H}^{(2)}}
\newcommand{\bI}{\mathop{}\!\mathsf{I}}

% Operateurs ----------------------------------------------
\newcommand{\di}[1]{\mathop{}\!\mathrm{d}#1}
\DeclareMathOperator{\OO}{\mathcal{O}}
\DeclareMathOperator{\oo}{\mathcal{\scriptstyle O}}
\DeclareMathOperator{\Div}{div}
\DeclareMathOperator{\Rot}{\mathbf{curl}}

% delimiters ----------------------------------------------
\newcommand{\plr}[1]{\left(#1\right)}
\newcommand{\clr}[1]{\left[#1\right]}
\newcommand{\abs}[1]{\left\lvert#1\right\rvert}
\newcommand{\norm}[1]{\left\lVert#1\right\rVert}

% Vecteurs ------------------------------------------------
\newcommand{\vect}[1]{\boldsymbol{#1}}
\newcommand{\vx}{\boldsymbol{x}}

% Keywords ------------------------------------------------
\newcommand{\eps}{\varepsilon}
\newcommand{\comp}{\mathrm{comp}}
\newcommand{\loc}{\mathrm{loc}}
\newcommand{\inc}{\mathsf{in}}
\newcommand{\sca}{\mathsf{sc}}
\newcommand{\ecav}{\varepsilon_\mathsf{c}}
\newcommand{\mcav}{\mu_\mathsf{c}}


% Titre ===================================================
\title{Scattering by rotationally invariant cavities}
\author{Zoïs \textsc{Moitier}}


%==========================================================
\begin{document}

\maketitle

\begin{abstract}
    We solve scattering problem by rotationally invariant cavities.
    The shape consider are disk and annulus where the permittivity and permeability are radial function.
\end{abstract}

\tableofcontents

%============================
\section{The scattering problem}
%============================

% = = = = = = = = = = = = = =
\subsection{Problem statement}
% = = = = = = = = = = = = = =

Let $\bbD = \{\vx \in \bbR^2 \mid |\vx| < 1\}$ be the unit disk and $\bbA_\delta = \{\vx \in \bbR^2 \mid \delta < |\vx| < 1\}$ an annulus of width $\delta > 0$.
The cavity $\Omega$ denote either $\bbD$ or $\bbA_\delta$ and the interface $\Gamma = \partial\Omega$ is the boundary of $\Omega$.
We define the function $\eps \in \rmL^\infty(\bbR^2)$ (permittivity) and $\mu \in \rmL^\infty(\bbR^2)$ (permeability) as
\[
    \eps(\vx) = \begin{dcases}
        \ecav(|\vx|) & \text{if } \vx \in \overline{\Omega}\\
        1 & \text{otherwise}
    \end{dcases} \qquad
    \text{and} \qquad
    \mu(\vx) = \begin{dcases}
        \mcav(|\vx|) & \text{if } \vx \in \overline{\Omega}\\
        1 & \text{otherwise}
    \end{dcases}
\]
where $\ecav, \mcav \in \scrC^\infty(\overline{I}, \bbR^*)$ with $I = (0,1)$ or $I = (\delta, 1)$.
Let $\calD^2 \coloneqq \{u \in \rmL^2(\bbR^2) \mid \Div(\eps^{-1} \nabla u) \in \rmL^2(\bbR^2)\}$ be the domain of the operator $u \mapsto -\mu^{-1}\Div(\eps^{-1} \nabla u)$ and we define the ``loc'' version
\[
    \calD_\loc^2 \coloneqq \{u \in \rmL_\loc^2(\bbR^2) \mid \forall \chi \in \scrC_\comp^\infty(\bbR^2),\ \chi u \in \calD^2\}.
\]

We define the following scattering problem: Given a wavenumber $k > 0$ and an incident field $u^\inc : \vx \mapsto \ex^{\im\, k\, y}$, find the scattering field $u^\sca \in \calD_\loc^2$ such that the total field $u = u^\inc + u^\sca$ satisfy
\begin{subequations}\label{prob:scattering}
\begin{equation}\label{eq:scat_pde}
    \begin{dcases}
        -\mu^{-1}\Div\plr{\eps^{-1} \nabla u} - k^2 u = 0 & \text{in } \Omega \text{ and } \bbR^2 \setminus \overline{\Omega}\\[1ex]
        \clr{u}_\Gamma = 0 \ \text{and}\ \clr{\eps^{-1} \partial_{\vect{\nu}} u}_\Gamma = 0 & \text{across } \Gamma\\[1ex]
        \text{$u^\sca$ is $k$-outgoing} 
    \end{dcases}
\end{equation}
where $\vect{\nu} : \Gamma \to \bbS^1$ is the exterior unit normal and $u^\sca$ is $k$-outgoing mean that for $|\vx| \ge 1$, there exist $(c_m)_{m \in \bbZ} \in \ell^2(\bbZ)$ such that
\begin{equation}\label{eq:OWC}
    u^\sca(\vx) = \sum_{m \in \bbZ} c_m\, \Hu_m(k\, r)\, \ex^{\im\, m\, \theta}
\end{equation}
with $(r, \theta) \in \bbR_+ \times \bbR / 2\pi\bbZ$ the polar coordinate associated to the Cartesian coordinates $\vx \in \bbR^2$ and $z \mapsto \Hu_m(z)$ is the Hankel function of the first kind of order $m$.
\end{subequations}

% = = = = = = = = = = = = = =
\subsection{1D reduction}
% = = = = = = = = = = = = = =

We look for solution of problem \eqref{prob:scattering} of the form
\[
    u(\vx) = \sum_{m \in \bbZ} w_m(r)\, \ex^{\im\, m\, \theta} \qquad
    \text{where } w_m(r) \coloneqq \frac{1}{2\pi} \int_0^{2\pi} u(r,\theta)\, \ex^{-\im\, m\, \theta} \di{\theta}.
\]
Similarly we write $u^\inc(\vx) = \sum_{m \in \bbZ} w_m^\inc(r)\, \ex^{\im\, m\, \theta}$ and $u^\sca(\vx) = \sum_{m \in \bbZ} w_m^\sca(r)\, \ex^{\im\, m\, \theta}$.
For the incident field, we have $w_m^\inc(r) = \bJ_m(k\, r)$ because the Jacobi-Anger expansion \cite[Eq.~10.12.1]{NIST:DLMF} states that
\begin{equation}\label{eq:JacobiAngerExpansion}
    u^\inc(\vx) = \ex^{\im k\, y} = \sum_{m \in \bbZ} \bJ_m(k\, r)\, \ex^{\im\, m\, \theta}
\end{equation}
where $z \mapsto \bJ_m(z)$ is the Bessel function of the first kind.
The series in equation \eqref{eq:JacobiAngerExpansion} converges absolutely on every compact set of $\bbR^2$.

The domain of the operator $w \mapsto -r^{-1}\, \mu^{-1} \partial_r(r\, \eps^{-1} \partial_r w) + m^2\, r^{-2}\, \eps^{-1}\mu^{-1} w$ and its ``loc'' version are define by
\begin{align*}
    \calD^{1, m} &\coloneqq \{w \in \rmL^2(\bbR_+^*, r\di{r}) \mid \partial_r(r\, \eps^{-1}\, \partial_r w) - m^2\, r^{-1}\eps^{-1} w \in \rmL^2(\bbR_+^*)\}\\
    \calD^{1, m}_\loc &\coloneqq \{w \in \rmL^2(\bbR_+^*, r\di{r}) \mid \forall \chi \in \scrC_\comp^\infty(\bbR),\ \chi w \in \calD^{1, m}\}.
\end{align*}

Problem \ref{prob:scattering} reduce to a family of problem index by $m \in \bbZ$: Given $k > 0$ find $w_m^\sca \in \calD^{1, m}_\loc$ such that $w_m = w_m^\inc + w_m^\sca$ and
\begin{equation}\label{prob:w_scattering}
     \begin{dcases}
        -\frac{1}{r\, \mu} \partial_r\plr{\frac{r}{\eps}\, \partial_r w_m} + \dfrac{m^2}{r^2\, \eps\mu} w_m - k^2\, w_m = 0 & \text{in } I \text{ and } \bbR^* \setminus \overline{I}      \\
        w_0'(0) = 0 & \text{on } \{0\}       \\
        \clr{w_m}_{\{1\}} = 0 \quad\text{and}\quad \clr{\eps^{-1}\, w_m'}_{\{1\}} = 0              & \text{across } \{1\}   \\
        w_m^\sca(r) \propto \Hu_m(k\, r)                                                              & r \ge 1
    \end{dcases}
\end{equation}
with $\propto$ meaning ``up to a multiplicative constant''.

%============================
\section{The resonances problem}
%============================

% = = = = = = = = = = = = = =
\subsection{Problem statement}
% = = = = = = = = = = = = = =

% = = = = = = = = = = = = = =
\subsection{1D reduction}
% = = = = = = = = = = = = = =

%============================
\section{Disk cavities with constant optical parameters}
%============================

% = = = = = = = = = = = = = =
\subsection{Solution of the scattering problem}
% = = = = = = = = = = = = = =

In this section, we assume that the cavity is the unit disk $\bbD$ and that $\ecav$ and $\mcav$ are non zero constant in $[0, 1]$.
The solution of problem \eqref{prob:w_scattering} for $0 < r < 1$ depend of the sign of $\ecav\mcav$.
We denote by $C_m$ the function
\[
    C_m(z) = \begin{dcases}
        \bJ_m(\sqrt{\ecav\mcav}\, z) & \text{if } \ecav\mcav > 0\\
        \bI_m(\sqrt{-\ecav\mcav}\, z) & \text{if } \ecav\mcav < 0
    \end{dcases}
\]
where $z \mapsto \bI_m(z)$ is the modified Bessel of the first kind.
The solution of problem \eqref{prob:w_scattering} is
\begin{equation}
    w_m(r) = \begin{dcases}
        \alpha_m\, C_m(k\, r) & \text{if } r \le 1\\
        \beta_m\, \Hu_m(k\, r) + \bJ_m(k\, r) & \text{if } r > 1
    \end{dcases}
\end{equation}
where $(\alpha_m, \beta_m)$ are solution of
\begin{equation}
    \begin{pmatrix*}[r]
        C_m(k) & -\Hu_m(k)\\[1ex]
        \ecav^{-1}\, C_m'(k) & -{\Hu_m}'(k)
    \end{pmatrix*}
    \begin{pmatrix}
        \alpha_m\\[1ex]
        \beta_m
    \end{pmatrix} = 
    \begin{pmatrix}
        \bJ_m(k)\\[1ex]
        \bJ_m'(k)
    \end{pmatrix}.
\end{equation}

% = = = = = = = = = = = = = =
\subsection{Solution of the resonance problem}
% = = = = = = = = = = = = = =

A resonances $\ell$ is a complex satisfying
\begin{equation}
    \ecav^{-1}\, C_m'(\ell) \Hu_m(\ell) - C_m(\ell) {\Hu_m}'(\ell) = 0
\end{equation}
with the associated mode
\begin{equation}
    w_\ell(r) = \begin{dcases}
        C_m(k\, r) & \text{if } r \le 1\\
        \tfrac{C_m(k)}{\Hu_m(k)}\Hu_m(k\, r) & \text{if } r > 1
    \end{dcases}.
\end{equation}

%============================
\section{Annulus cavities}
%============================

% = = = = = = = = = = = = = =
\subsection{Constant optical parameters}
% = = = = = = = = = = = = = =

% = = = = = = = = = = = = = =
\subsection{The flat well version}
% = = = = = = = = = = = = = =

% = = = = = = = = = = = = = =
\subsection{The full flat version}
% = = = = = = = = = = = = = =

\bibliographystyle{alpha}
\bibliography{references}

\end{document}
