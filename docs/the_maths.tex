% arara: indent: {overwrite: yes, modifylinebreaks: yes}
% arara: pdflatex: {draft: yes, shell : yes}
% arara: bibtex
% arara: pdflatex: {draft: yes, shell : yes}
% arara: pdflatex: {shell : yes}
% arara: clean: { extensions: [ log ], files: [ indent ] }
% arara: clean: { extensions: [ aux, bak, log, toc, out ]}

\documentclass[12pt,a4paper]{article}


% Usepackages =============================================
% Standard et geometrie -----------------------------------
\usepackage[utf8]{inputenc}
\usepackage[T1]{fontenc}
\usepackage[english]{babel}
\usepackage[margin=2cm]{geometry}

% Maths ---------------------------------------------------
\usepackage{amsmath,amssymb,amsfonts,amsthm}
\usepackage{mathtools,mathrsfs}

%Liens et dessins -----------------------------------------
\usepackage[colorlinks]{hyperref}
\usepackage{tikz}


% Couleurs ================================================
% definecolor ---------------------------------------------
\definecolor{coGB}{HTML}{1F77B4}
\definecolor{coOr}{HTML}{FF7F0E}
\definecolor{coWG}{HTML}{2CA02C}
\definecolor{coAR}{HTML}{D62728}
\definecolor{cMBP}{HTML}{9467BD}
\definecolor{coSM}{HTML}{8C564B}
\definecolor{coPP}{HTML}{E377C2}
\definecolor{coTG}{HTML}{7F7F7F}
\definecolor{coAG}{HTML}{BCBD22}
\definecolor{coBB}{HTML}{17BECF}
% hypersetup ----------------------------------------------
\hypersetup{
    colorlinks = True,
    linkcolor  = coAR,
    citecolor  = coWG,
    urlcolor   = coGB,
}


% Maths ===================================================
% mathbb --------------------------------------------------
\newcommand{\bbA}{\mathbb{A}}
\newcommand{\bbB}{\mathbb{B}}
\newcommand{\bbC}{\mathbb{C}}
\newcommand{\bbD}{\mathbb{D}}
\newcommand{\bbN}{\mathbb{N}}
\newcommand{\bbR}{\mathbb{R}}
\newcommand{\bbS}{\mathbb{S}}
\newcommand{\bbT}{\mathbb{T}}
\newcommand{\bbZ}{\mathbb{Z}}

% mathcal --------------------------------------------------
\newcommand{\calD}{\mathcal{D}}
\newcommand{\calN}{\mathcal{N}}
\newcommand{\calP}{\mathcal{P}}
\newcommand{\calV}{\mathcal{V}}

% mathscr --------------------------------------------------
\newcommand{\scrC}{\mathscr{C}}

% mathrm --------------------------------------------------
\newcommand{\rmL}{\mathrm{L}}
\newcommand{\rmH}{\mathrm{H}}

% Constants -----------------------------------------------
\newcommand{\ex}{\mathsf{e}}
\newcommand{\im}{\mathsf{i}}

% Fontions ------------------------------------------------
\newcommand{\bJ}{\mathop{}\!\mathsf{J}}
\newcommand{\bY}{\mathop{}\!\mathsf{Y}}
\newcommand{\Hu}{\mathop{}\!\mathsf{H}^{(1)}}
\newcommand{\Hd}{\mathop{}\!\mathsf{H}^{(2)}}
\newcommand{\bI}{\mathop{}\!\mathsf{I}}
\newcommand{\bK}{\mathop{}\!\mathsf{K}}

\newcommand{\bj}{\mathop{}\!\mathsf{j}}
\newcommand{\by}{\mathop{}\!\mathsf{y}}
\newcommand{\hu}{\mathop{}\!\mathsf{h}^{(1)}}
\newcommand{\hd}{\mathop{}\!\mathsf{h}^{(2)}}
\newcommand{\bi}{\mathop{}\!\mathsf{i}}
\newcommand{\bk}{\mathop{}\!\mathsf{k}}

\newcommand{\outh}{\mathop{}\!\mathfrak{h}^{(1)}}

% Operateurs ----------------------------------------------
\newcommand{\di}[1]{\mathop{}\!\mathrm{d}#1}
\DeclareMathOperator{\OO}{\mathcal{O}}
\DeclareMathOperator{\oo}{\mathcal{\scriptstyle O}}
\DeclareMathOperator{\Div}{div}
\DeclareMathOperator{\Rot}{\mathbf{curl}}
\newcommand{\Ind}[1]{\mathop{}\!\mathbf{1}_{#1}}

% delimiters ----------------------------------------------
\newcommand{\plr}[1]{\left(#1\right)}
\newcommand{\clr}[1]{\left[#1\right]}
\newcommand{\abs}[1]{\left\lvert#1\right\rvert}
\newcommand{\norm}[1]{\left\lVert#1\right\rVert}

% Vecteurs ------------------------------------------------
\newcommand{\vect}[1]{\boldsymbol{#1}}
\newcommand{\vx}{\boldsymbol{x}}

% Keywords ------------------------------------------------
\newcommand{\eps}{\varepsilon}
\newcommand{\comp}{\mathrm{comp}}
\newcommand{\loc}{\mathrm{loc}}
\newcommand{\inc}{\mathsf{in}}
\newcommand{\sca}{\mathsf{sc}}
\newcommand{\ecav}{\varepsilon_\mathsf{c}}
\newcommand{\mcav}{\mu_\mathsf{c}}


% Titre ===================================================
\title{claudius: analytic computations of scattering \\ \Large \emph{calculs analytiques pour la diffusion}}
\author{Zoïs \textsc{Moitier}}

%==========================================================
\begin{document}

\maketitle

\begin{abstract}
    We solve scattering problem for cases where we have analytical expressions.
    All cavities consider will be invariant by rotation.
    The shape consider are disk, annulus, ball and spherical shell where the permittivity and permeability are radial function.
\end{abstract}

\tableofcontents

%=========================
\section{General settings}
%=========================

We consider the scattering by spherical cavities, disk or annulus in dimension $2$, and ball or spherical shell in dimension $3$.
We call $\abs{\,\cdot\,}_d$ the euclidean norm in dimension $d = 2, 3$.
Disk and ball are denoted by $\bbB_\rho^d = \{\vx \in \bbR^d \mid \abs{\vx}_d < \rho\}$ for $\rho > 0$, and annulus and spherical shell are denoted by $\bbA_{\rho, \sigma}^d = \{\vx \in \bbR^d \mid \rho < \abs{\vx}_d < \sigma\}$ for $0 < \rho < \sigma$.
The cavities will be compose of concentric spherical shell with impenetrable or penetrable core and denoted by $\Omega$.
For a cavity with impenetrable core and $N \ge 0$ layers, we have $\Omega = \bbA_{\rho_1, \rho_2}^d \cup \bbA_{\rho_2, \rho_3}^d \cup \cdots \cup \bbA_{\rho_N, \rho_{N+1}}^d$ with $0 < \rho_1 < \rho_2 < \cdots < \rho_{N+1}$.
For a cavity with penetrable core and $N \ge 0$ layers, we have $\Omega = \bbB_{\rho_1}^d \cup \bbA_{\rho_1, \rho_2}^d \cup \cdots \cup \bbA_{\rho_N, \rho_{N+1}}^d$ with $0 = \rho_0 < \rho_1 < \cdots < \rho_{N+1}$.
The boundary/interfaces $\Gamma$ of the different layers of the cavity $\Omega$ is compose of concentric spheres $\Gamma = \rho_1\bbS^{d-1} \cup \rho_2\bbS^{d-1} \cup \cdots \cup \rho_N\bbS^{d-1}$.

\bigskip

We define the function $\eps \in \rmL^\infty(\bbR^d)$ and $\mu \in \rmL^\infty(\bbR^d)$ as
\[
    \eps(\vx) = \begin{dcases}
        \ecav\plr{\abs{\vx}_d} & \text{if } \vx \in \Omega\\
        1 & \text{otherwise}
    \end{dcases} \qquad
    \text{and} \qquad
    \mu(\vx) = \begin{dcases}
        \mcav\plr{\abs{\vx}_d} & \text{if } \vx \in \Omega\\
        1 & \text{otherwise}
    \end{dcases}
\]
where
\[
    \ecav = \sum_{n = 0}^{N} \eps_n \Ind{(\rho_n, \rho_{n+1})} \qquad
    \text{and} \qquad
    \mcav = \sum_{n = 0}^{N} \mu_n \Ind{(\rho_n, \rho_{n+1})}
\]
with $\eps_n, \mu_n \in \scrC^\infty([\rho_n, \rho_{n+1}], \bbR^*)$ for $0 \le n \le N$.
For the impenetrable case the index $n = 0$ is not use.
In some context the function can be view as the permittivity ($\eps$) and the permeability ($\mu$) of a material.

\bigskip

A positive wavenumber is noted $k$.
We only consider a plane wave incident field, we have $u^\inc : \vx \mapsto \exp(\im\, k\, \vect{\nu} \cdot \vx)$ with $\vect{\nu} \in \bbS^{d-1}$ the direction of the plane wave.
In the following, the scattered field is noted $u^\sca$ and the total field is noted $u$.

%=============================
\section{Helmholtz's equation}
%=============================

%----------------------------------
\subsection{The scattering problem}
%----------------------------------

We consider the operator $u \mapsto -\mu^{-1}\Div(\eps^{-1}\, \nabla u)$ of domain
\[
    D^{d, \eps} \coloneqq \{u \in \rmL^2(\bbR^d) \mid \Div(\eps^{-1}\, \nabla u) \in \rmL^2(\bbR^d)\}
\]
and we define the ``locale'' version
\[
    D_\loc^{d, \eps} \coloneqq \{u \in \rmL_\loc^2(\bbR^d) \mid \forall \chi \in \scrC_\comp^\infty(\bbR^d),\ \chi u \in D^{d, \eps}\}.
\]

\bigskip

We define the following scattering problem: Given a wavenumber $k > 0$ and an incident field $u^\inc$, find the scattering field $u^\sca \in D_\loc^{d, \eps}$ such that the total field $u = u^\inc + u^\sca$ satisfy
\begin{equation}\label{eq:scat_pde_im}
    (\calD / \calN)\begin{dcases}
        -\mu^{-1}\Div\plr{\eps^{-1} \nabla u} - k^2 u = 0 & \text{in } \Omega \text{ and } \bbR^2 \setminus \overline{\Omega}\\[1ex]
        u = 0 \qquad \text{or} \qquad \partial_{\vect{n}}u = 0 & \text{on } \rho_1\bbS^{d-1}\\[1ex]
        \clr{u} = 0 \ \text{and}\ \clr{\eps^{-1} \partial_{\vect{n}} u} = 0 & \text{across } \rho_2\bbS^{d-1} \cup \cdots \cup \rho_N\bbS^{d-1}\\[1ex]
        \text{$u^\sca$ is $k$-outgoing} 
    \end{dcases}
\end{equation}
in the Dirichlet ($\calD$) or Neumann ($\calN$) case, or
\begin{equation}\label{eq:scat_pde}
    (\calP)\begin{dcases}
        -\mu^{-1}\Div\plr{\eps^{-1} \nabla u} - k^2 u = 0 & \text{in } \Omega \text{ and } \bbR^2 \setminus \overline{\Omega}\\[1ex]
        \clr{u} = 0 \ \text{and}\ \clr{\eps^{-1} \partial_{\vect{n}} u} = 0 & \text{across } \rho_1\bbS^{d-1} \cup \cdots \cup \rho_N\bbS^{d-1}\\[1ex]
        \text{$u^\sca$ is $k$-outgoing} 
    \end{dcases}
\end{equation}
in the penetrable case, where $\vect{n} : \Gamma \to \bbS^{d-1}$ are the outward unit normal and $u^\sca$ is $k$-outgoing mean that for $|\vx| \ge \rho_N$, there exist $\beta$ such that
\begin{subequations}\label{eq:OWC}
\begin{align}
    u^\sca(\vx) &= \sum_{m \in \bbZ} \beta_m\, \Hu_m(k\, r)\, \ex^{\im\, m\, \theta} && d = 2\\
    u^\sca(\vx) &= \sum_{\ell = 0}^{+\infty} \sum_{m = -\ell}^\ell \beta_\ell^m\, \hu_\ell(k\, r)\, Y_\ell^m(\theta,\phi) && d = 3
\end{align}
\end{subequations}
with $(r, \theta) \in \bbR_+ \times \bbR / 2\pi\bbZ$ the polar coordinate, 
$(r, \theta, \phi) \in \bbR_+ \times [0, \pi] \times \bbR / 2\pi\bbZ$ the spherical coordinate, $Y_\ell^m$ the spherical harmonic, and $z \mapsto \Hu_m(z)$ (resp.\ $z \mapsto \hu_m(z)$) is the cylindrical (resp.\ spherical) Hankel function of the first kind of order $m$.

%----------------------------
\subsection{The 1d reduction}
%----------------------------

%===========================
\section{Maxwell's equation}
%===========================

\appendix

%======================
\section{Miscellaneous}
%======================

%-----------------------
\subsection{Coordinates}
%-----------------------

%
\subsubsection{Polar}
%

In dimension two, the Cartesian coordinate $\vx = (x, y) \in \bbR^2$ are describe in term of polar coordinate $(r, \theta) \in \bbR_+ \times \bbR / 2\pi\bbZ$ by
\[
    \begin{dcases}
        x = r \cos(\theta)\\
        y = r \sin(\theta)
    \end{dcases} \qquad
    \text{and} \qquad
    \begin{dcases}
        r = \sqrt{x^2 + y^2}\\
        \theta = \arg(x + \im y)
    \end{dcases}
\]

%
\subsubsection{Spherical}
%

In dimension three, the Cartesian coordinate $\vx = (x, y, z) \in \bbR^3$ are describe in term of spherical coordinate $(r, \theta, \phi) \in \bbR_+ \times [0, \pi] \times \bbR / 2\pi\bbZ$ by
\[
    \begin{dcases}
        x = r \sin(\theta) \cos(\phi)\\
        y = r \sin(\theta) \sin(\phi)\\
        z = r \cos(\theta)
    \end{dcases} \qquad
    \text{and} \qquad
    \begin{dcases}
        r = \sqrt{x^2 + y^2 + z^2}\\
        \theta = \arccos\plr{\frac{z}{r}}\\
        \phi = \arg(x + \im y)
    \end{dcases}.
\]

%------------------------------
\subsection{Bessel's functions}
%------------------------------

%
\subsubsection{Cylindrical}
%

The cylindrical Bessel's function $\bJ_m$, $\bY_m$, $\Hu_m$, and $\Hd_m$ are define in \cite[Sec.~10.2]{NIST:DLMF} and they are solutions of the ODE
\[
    -\frac{1}{z}\plr{z\, w'}' + \frac{m^2}{z^2}w - w = 0.
\]
The cylindrical modified Bessel's function $\bI_m$ and $\bK_m$ are define in \cite[Sec.~10.25]{NIST:DLMF} and they are solutions of the ODE
\[
    -\frac{1}{z}\plr{z\, w'}' + \frac{m^2}{z^2}w + w = 0.
\]

%
\subsubsection{Spherical}
%

The spherical Bessel's function $\bj_m$, $\by_m$, $\hu_m$, $\hd_m$ ($-$) and $\bi_m$, $\bk_m$ ($+$) are define in \cite[Sec.~10.47]{NIST:DLMF} and they are solutions of the ODE
\[
    -\frac{1}{z^2}\plr{z^2\, w'}' + \frac{\ell(\ell+1)}{z^2}w \mp w = 0
\]
We have the following relation between the cylindrical and spherical Bessel function
\[
    \mathsf{f}_\ell(z) = \sqrt{\frac{\pi}{2z}} \mathsf{F}_{\ell+\frac{1}{2}}(z)
\]
where the couples $(\mathsf{f}, \mathsf{F})$ are $(\bj, \bJ)$, $(\by, \bY)$, $(\hu, \Hu)$, $(\hd, \Hd)$, $(\bi, \bI)$, and $(\bk, \bK)$.

%------------------------------
\subsection{Spherical harmonic}
%------------------------------

For $\ell \in \bbN$ and $m \in \{-\ell, \ldots, 0, \ldots, \ell\}$, the spherical harmonic $Y_\ell^m$ in the spherical coordinate is define by
\[
    Y_\ell^m(\theta, \phi) = (-1)^m\, \sqrt{\frac{2\ell+1}{4\pi}\, \frac{(\ell-m)!}{(\ell+m)!}}\, P_\ell^m(\cos(\theta))\, \ex^{\im\, m\, \phi}
\]
where $P_\ell^m$ are the associated Legendre polynomial define by
\[
    P_\ell^m(x) = \frac{1}{2^\ell\, \ell!}\, (1-x^2)^{\frac{m}{2}}\, \partial_x^{\ell+m} (x^2-1)^\ell.
\]

%--------------------------------------------------
\subsection{Spherical wave expansion of plane wave}
%--------------------------------------------------

The Jacobi-Anger expansion \cite[Eq.~10.12.1]{NIST:DLMF} states that, for $z \in \bbC$ and $t \in \bbC^*$,
\begin{equation}\label{eq:JacobiAngerExpansion}
    \ex^{\frac{z}{2}(t-t^{-1})} = \sum_{m \in \bbZ} \bJ_m(z)\, t^m.
\end{equation}
Which give the following expansion for a 2d plane wave
\begin{equation}
    \ex^{\im\, k\, y} = \sum_{m \in \bbZ} \bJ_m(k\, r)\, \ex^{\im\, m\, \theta}
\end{equation}
where $(x, y)$ are the Cartesian coordinates and $(r, \theta)$ the corresponding polar coordinates.

We have the following expansion for a 3d plane wave
\begin{equation}
    \ex^{\im\, k\, \vect{\nu} \cdot \vx} = 4\pi \sum_{\ell = 0}^{+\infty} \im^\ell\, \bj_\ell(k\, r) \sum_{m = -\ell}^\ell Y_\ell^m(\vect{\nu}) \overline{Y_\ell^m(\omega)}
\end{equation}
where $(x, y)$ are the Cartesian coordinates and $(r, \theta)$ the corresponding polar coordinates.


\bibliographystyle{alpha}
\bibliography{references}


%##############################################################################
\newpage
%##############################################################################

%================================
\section{The scattering problems}
%================================

% = = = = = = = = = = = = = =
\subsection{1D reduction}
% = = = = = = = = = = = = = =

We look for solution of problem of the form
\[
    u(\vx) = \sum_{m \in \bbZ} w_m(r)\, \ex^{\im\, m\, \theta} \qquad
    \text{where } w_m(r) \coloneqq \frac{1}{2\pi} \int_0^{2\pi} u(r,\theta)\, \ex^{-\im\, m\, \theta} \di{\theta}.
\]
Similarly we write $u^\inc(\vx) = \sum_{m \in \bbZ} w_m^\inc(r)\, \ex^{\im\, m\, \theta}$ and $u^\sca(\vx) = \sum_{m \in \bbZ} w_m^\sca(r)\, \ex^{\im\, m\, \theta}$.
The series in equation \eqref{eq:JacobiAngerExpansion} converges absolutely on every compact set of $\bbR^2$.

The domain of the operator $w \mapsto -r^{-1}\, \mu^{-1} \partial_r(r\, \eps^{-1} \partial_r w) + m^2\, r^{-2}\, \eps^{-1}\mu^{-1} w$ and its ``loc'' version are define by
\begin{align*}
    \calD^{1, m} &\coloneqq \{w \in \rmL^2(\bbR_+^*, r\di{r}) \mid \partial_r(r\, \eps^{-1}\, \partial_r w) - m^2\, r^{-1}\eps^{-1} w \in \rmL^2(\bbR_+^*)\}\\
    \calD^{1, m}_\loc &\coloneqq \{w \in \rmL^2(\bbR_+^*, r\di{r}) \mid \forall \chi \in \scrC_\comp^\infty(\bbR),\ \chi w \in \calD^{1, m}\}.
\end{align*}

Problem reduce to a family of problem index by $m \in \bbZ$: Given $k > 0$ find $w_m^\sca \in \calD^{1, m}_\loc$ such that $w_m = w_m^\inc + w_m^\sca$ and
\begin{equation}\label{prob:w_scattering}
     \begin{dcases}
        -\frac{1}{r\, \mu} \partial_r\plr{\frac{r}{\eps}\, \partial_r w_m} + \dfrac{m^2}{r^2\, \eps\mu} w_m - k^2\, w_m = 0 & \text{in } I \text{ and } \bbR^* \setminus \overline{I}      \\
        w_0'(0) = 0 & \text{on } \{0\}       \\
        \clr{w_m}_{\{1\}} = 0 \quad\text{and}\quad \clr{\eps^{-1}\, w_m'}_{\{1\}} = 0              & \text{across } \{1\}   \\
        w_m^\sca(r) \propto \Hu_m(k\, r)                                                              & r \ge 1
    \end{dcases}
\end{equation}
with $\propto$ meaning ``up to a multiplicative constant''.

%============================
\section{The resonances problem}
%============================

% = = = = = = = = = = = = = =
\subsection{Problem statement}
% = = = = = = = = = = = = = =

% = = = = = = = = = = = = = =
\subsection{1D reduction}
% = = = = = = = = = = = = = =

%============================
\section{Disk cavities with constant optical parameters}
%============================

\paragraph{Solution of the scattering problem.}
In this section, we assume that the cavity is the unit disk $\bbD$ and that $\ecav$ and $\mcav$ are non zero constant in $[0, 1]$.
The solution of problem \eqref{prob:w_scattering} for $0 < r < 1$ depend of the sign of $\ecav\mcav$.
We denote by $C_m$ the function
\[
    C_m(z) = \begin{dcases}
        \bJ_m(\sqrt{\ecav\mcav}\, z) & \text{if } \Re(\ecav\mcav) > 0\\
        \bI_m(\sqrt{-\ecav\mcav}\, z) & \text{if } \Re(\ecav\mcav) < 0
    \end{dcases}
\]
where $z \mapsto \bI_m(z)$ is the modified Bessel of the first kind.
The solution of problem \eqref{prob:w_scattering} is
\begin{equation}
    w_m(r) = \begin{dcases}
        a_m\, C_m(k\, r) & \text{if } r \le 1\\
        b_m\, \Hu_m(k\, r) + \bJ_m(k\, r) & \text{if } r > 1
    \end{dcases}
\end{equation}
where $(a_m, b_m)$ are solution of
\begin{equation}
    \begin{pmatrix*}[r]
        C_m(k) & -\Hu_m(k)\\[1ex]
        \ecav^{-1}\, C_m'(k) & -{\Hu_m}'(k)
    \end{pmatrix*}
    \begin{pmatrix}
        a_m\\[1ex]
        b_m
    \end{pmatrix} = 
    \begin{pmatrix}
        \bJ_m(k)\\[1ex]
        \bJ_m'(k)
    \end{pmatrix}.
\end{equation}


\paragraph{Solution of the resonance problem.}
A resonances $\ell$ is a complex satisfying
\begin{equation}
    \ecav^{-1}\, C_m'(\ell) \Hu_m(\ell) - C_m(\ell) {\Hu_m}'(\ell) = 0
\end{equation}
with the associated mode
\begin{equation}
    w_\ell(r) = \begin{dcases}
        C_m(k\, r) & \text{if } r \le 1\\
        \tfrac{C_m(k)}{\Hu_m(k)}\Hu_m(k\, r) & \text{if } r > 1
    \end{dcases}.
\end{equation}

%============================
\section{Annulus cavities}
%============================

\paragraph{Solution of the scattering problem.}
In this section, we assume that the cavity is the annulus $\bbA_\delta = \{\vx \in \bbR^2 \mid \delta < |\vx| < 1\}$ with $0 < \delta < 1$ and that $\ecav$ and $\mcav$ are non zero constant in $[0, 1]$.
The solution of problem \eqref{prob:w_scattering} for $\delta < r < 1$ depend of the sign of $\ecav\mcav$ and are denoted by $C_m$ and $D_m$.
The solution of problem \eqref{prob:w_scattering} is
\begin{equation}
    w_m(r) = \begin{dcases}
        a_m\, \bJ_m(k\, r) & \text{if } 0 \le r < \delta\\
        b_m\, C_m(k\, r) + c_m\, D_m(k\, r) & \text{if } \delta \le r \le 1\\
        d_m\, \Hu_m(k\, r) + \bJ_m(k\, r) & \text{if } r > 1
    \end{dcases}
\end{equation}
where $(a_m, b_m, c_m, d_m)$ are solution of
\begin{equation}
    \begin{pmatrix*}[r]
        -\bJ_m(k\delta) & C_m(k\delta) & D_m(k\delta) & 0\\[1ex]
        -\bJ_m'(k\delta) & \ecav^{-1}\, C_m'(k\delta) & \ecav^{-1}\, D_m'(k\delta) & 0\\[1ex]
        0 & C_m(k) & D_m(k) & -\Hu_m(k)\\[1ex]
        0 & \ecav^{-1}\, C_m'(k) & \ecav^{-1}\, D_m'(k) & -{\Hu_m}'(k)
    \end{pmatrix*}
    \begin{pmatrix}
        a_m\\[1ex]
        b_m\\[1ex]
        c_m\\[1ex]
        d_m
    \end{pmatrix} = 
    \begin{pmatrix}
        0\\[1ex]
        0\\[1ex]
        \bJ_m(k)\\[1ex]
        \bJ_m'(k)
    \end{pmatrix}.
\end{equation}


\paragraph{Solution of the resonance problem.}
A resonances $\ell$ is a complex satisfying
\begin{equation}
    \ecav^{-1}\, C_m'(\ell) \Hu_m(\ell) - C_m(\ell) {\Hu_m}'(\ell) = 0
\end{equation}
with the associated mode
\begin{equation}
    w_\ell(r) = \begin{dcases}
        C_m(k\, r) & \text{if } r \le 1\\
        \tfrac{C_m(k)}{\Hu_m(k)}\Hu_m(k\, r) & \text{if } r > 1
    \end{dcases}.
\end{equation}

% = = = = = = = = = = = = = =
\subsection{Constant optical parameters}
% = = = = = = = = = = = = = =

\[
    C_m(z) = \begin{dcases}
        \bJ_m(\sqrt{\ecav\mcav}\, z) & \text{if } \Re(\ecav\mcav) > 0\\
        \bI_m(\sqrt{-\ecav\mcav}\, z) & \text{if } \Re(\ecav\mcav) < 0
    \end{dcases} \quad
    \text{and} \quad
    D_m(z) = \begin{dcases}
        \bY_m(\sqrt{\ecav\mcav}\, z) & \text{if } \Re(\ecav\mcav) > 0\\
        \bK_m(\sqrt{-\ecav\mcav}\, z) & \text{if } \Re(\ecav\mcav) < 0
    \end{dcases}
\]
where $z \mapsto \bK_m(z)$ is the modified Bessel of the second kind.

% = = = = = = = = = = = = = =
\subsection{The flat well version}
% = = = = = = = = = = = = = =

\end{document}
